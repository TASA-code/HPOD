

\section{Introductory on Astrodynamics}
This section will be focusing on the fundamental theory and concept of astrodynamics, starting from coordinate reference frames, orbital elements, to relating time with satellite orbit position. 
\vspace{0.3cm}
\hrule\vspace{0.1cm}
\hrule
\vspace{0.3cm}
\begin{multicols}{2}


\subsection{Reference Frame - Earth Based System}
One of the fundamental requirements of describing an orbit is to select a suitable reference frame. Beginning January 1, 1998, the IAU adopted the \textbf{\textit{International Celestial Reference System}} (ICRS) as the standard reference system, introducing the ICRF frame where the origin is at the barycentre of the solar system. In 1991, the IAU introduced the Barycentric (BCRS) and Geocentric (GCRS) Celestial Reference Systems. The IAU-2000 Resolutions use both the celestial (GCRF) and terrestrial (ITRF) frames, where the origin references the Earth's centre.

\subsubsection{GCRF}

\subsubsection{ITRF}

\subsubsection{PQW}

\subsubsection{LVLH}

\subsection{Defining an Orbit - Orbit Elements}

\subsection{Position and Velocity}


\end{multicols}
